\documentclass{article}

% Language and font encoding
\usepackage[english]{babel}
\usepackage[utf8]{inputenc}
\usepackage[T1]{fontenc}

% Math packages
\usepackage{amsmath}
\usepackage{amsfonts}
\usepackage{amssymb}
\usepackage{graphicx}

\title{Harmonic Oscillator Notes: Information Erasure and Phase Space Geometry}
\author{Gemini AI Assistant}

\begin{document}

\maketitle

\section{Information Theory and Erasure Action}

The fundamental bound on information erasure, derived from the Landauer principle and the energy-time uncertainty relation, is given by:
\begin{equation}
    b E \tau \le \pi \hbar
\end{equation}
where $b$ is the count of bits erased, $E$ is the erasure energy ($k_B T \ln 2$), and $\tau$ is the erasure time. This product defines the erasure action $A$. For a system in phase space, we split this action into coordinate and momentum components:
\begin{equation}
    (b_q + b_p) E \tau \le A_q + A_p \le \pi \hbar
\end{equation}

\section{Geometric Constraints of the Oscillator}

For the harmonic oscillator, emerging from the potential and kinetic components of the Hamiltonian $H=V+T$, we define a conjugate pair of action ellipses. 

\subsection{Ellipses vs. Rectangles}
We must distinguish between the bounding box (rectangle) and the actual action (elliptical area). Let $\Delta q$ and $\Delta p$ be the \textbf{full widths (diameters)} of the erasure region. The action area $A_q$ is defined by the ellipse inscribed in the rectangle formed by the erasure width $\delta_q$ and the system width $\Delta p$:
\begin{equation}
    A_q = \frac{\pi}{4} \delta_q \Delta p \le \frac{\pi \hbar}{2}
\end{equation}

\subsection{Hierarchy of Uncertainty}
The erasure distance $\epsilon$ is the distance between the observable state $q$ and the unknowable microstate $\mu$. This distance is bounded by the action quantum width $\delta$, which is in turn bounded by the conjugate system diameter $\Delta$:
\begin{equation}
    \epsilon_q = |\mu_q - q| \le \delta_q \le \Delta q
\end{equation}

\section{Dimensionless Normalization}

To facilitate simulation, we switch to dimensionless units using characteristic scales $q_0$ and $p_0$. We define $\mathcal{Q}$ and $\mathcal{P}$ as the \textbf{dimensionless radii} (semi-axes) of the system.

\subsection{Variable Definitions}
The diameters (full widths) are related to the dimensionless radii as follows:
\begin{align}
    \Delta q &= q_0 (2\mathcal{Q}) \\
    \Delta p &= p_0 (2\mathcal{P})
\end{align}

We define the characteristic scale such that the product $q_0 p_0$ equals the phase space quantum $h$:
\begin{equation}
    h = q_0 p_0 = 2\pi\hbar
\end{equation}

\subsection{Deriving the Dimensionless Bound}
Substituting the radius definitions into the action limit (Eq. 3):
\begin{equation}
    \frac{\pi}{4} q_0 |q_{?} - \mathbf{q}^*| p_0 (2\mathcal{P}) \le \frac{\pi \hbar}{2}
\end{equation}
Substituting $q_0 p_0 = 2\pi\hbar$:
\begin{equation}
    \frac{\pi}{4} (2\pi\hbar) |q_{?} - \mathbf{q}^*| (2\mathcal{P}) \le \frac{\pi \hbar}{2}
\end{equation}
\begin{equation}
    \pi^2 \hbar |q_{?} - \mathbf{q}^*| \mathcal{P} \le \frac{\pi \hbar}{2}
\end{equation}
Dividing by $\pi^2\hbar$ yields the normalized erasure bound:
\begin{equation}
    \epsilon_q^* = |q_{?} - \mathbf{q}^*| \le \frac{1}{2\pi \mathcal{P}}
\end{equation}

\section{Simulation Analysis and Physical Recovery}



In the simulation, histograms of erasure occupancy revealed a discrete node count $n$. A parabolic fit was observed between the node count and the coordinate.

\subsection{The Unit Problem: Area vs. Circumference}
Action $A$ has units of area. In dimensionless units, the area of our phase space orbit is:
\begin{equation}
    \text{Area} = \pi \mathcal{Q} \mathcal{P}
\end{equation}
In the quantum limit, the number of nodes $n$ corresponds to the number of action quanta $h$ enclosed. Assuming a symmetric oscillator ($\mathcal{Q} = \mathcal{P}$):
\begin{equation}
    n = \pi \mathcal{Q}^2 \implies \mathcal{Q} = \sqrt{\frac{n}{\pi}}
\end{equation}

The empirical fit used in the plotting code, $p_{val} = 2\pi\sqrt{n}$, suggests that the plotted coordinate $Q$ corresponds to the **circumference** ($2\pi \times \text{radius}$) of a circle in phase space. This transformation links the topological periodicity of the erasure cycle directly to the energy levels of the harmonic oscillator:
\begin{equation}
    n \propto \mathcal{Q}^2
\end{equation}
This confirms that the information-theoretic constraint $\epsilon \le \frac{1}{2\pi \mathcal{P}}$ successfully reconstructs the expected quantum mechanical scaling where Energy (and thus node count) is proportional to the square of the amplitude.

\end{document}