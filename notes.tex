\documentclass[11pt]{article}

% --- Essential Packages ---
\usepackage[utf8]{inputenc}
\usepackage[margin=1in]{geometry}
\usepackage{amsmath, amssymb} % For advanced math and subequations
\usepackage{xcolor}           % For colored boxes in lab notes
\usepackage{hyperref}         % For clickable references

% --- Page Style ---
\pagestyle{headings}
\title{Notes on Information Erasure and Fine Structure}
\author{Research Draft Summary}
\date{\today}

\begin{document}

\maketitle

\section{The Information-Action Limit}
The available action for erasure inside de Gosson's action quanta is defined by:
\begin{equation}
\label{eq:InfoActionLimit}
b E \tau \le \pi \hbar 
\end{equation}
where $b$ is the bit count, $E$ is Landauer erasure energy, and $\pi \hbar$ represents the minimal elliptical phase-space area.

\section{Erasure in Conjugate Action Quanta}
Erasure within a conjugate pair of Heisenberg cells is bounded as follows:
\begin{subequations}
\label{eq:SymplecticBounds}
\begin{equation}
\label{eq:SymplecticBoundsq}
\frac{\pi}{4}|\mu_q - q|\Delta p \le \frac{\pi \hbar}{2} 
\end{equation}
\begin{equation}
\label{eq:SymplecticBoundsp}
\frac{\pi}{4}|\mu_p - p|\Delta q \le \frac{\pi \hbar}{2} 
\end{equation}
\end{subequations}
Each coordinate displacement $(\mu_q - q)$ utilizes exactly half of the available de Gosson action.

\section{Fine Structure Scaling}
The fine structure constant $\alpha$ emerges from the geometric scaling of these boundaries:
\begin{equation}
\alpha = \frac{|\delta_q| \Delta p}{2 \hbar} = \frac{2 \hbar}{\Delta q \Delta p}
\end{equation}
The global scaling between the erasure shift and the system boundary follows the squared relation:
\begin{equation}
\alpha^2 = \frac{\delta_q}{\Delta q}
\end{equation}

\vspace{2em}

% --- Lab Notebook Summary Box ---
\begin{center}
\fcolorbox{black}{gray!10}{\begin{minipage}{0.9\textwidth}
\textbf{Lab Note: Geometric Derivation of $\alpha$ via de Gosson Quanta} \\
\textit{Core Concept:} $\alpha$ scales the Landauer erasure distance to the hard Heisenberg boundaries of the de Gosson "quantum blob."

\begin{itemize}
    \item \textbf{Action Partition:} Each erasure utilizes exactly $\pi\hbar/2$ of the conjugate pair. 
    \item \textbf{Hamiltonian Scaling:} For $\alpha \approx 1/137$, the total support $\Delta q \Delta p$ encompasses $\alpha^{-1}$ units of the action bound ($2\hbar$). 
    \item \textbf{Mappings:} $\delta_q \to r_e$ (Classical Radius), $\Delta q \to a_0$ (Bohr Radius).
    \item \textbf{Information Link:} Bit complexity $b \propto \alpha^{-1}$, suggesting stability is reached at the Landauer limit across exactly 137 action quanta.
\end{itemize}
\end{minipage}}
\end{center}


\subsection*{Historical Context: From Sommerfeld to Information Erasure}

Originally introduced by \textbf{Arnold Sommerfeld} in 1916, the fine-structure constant $\alpha \approx 1/137$ was defined as the ratio of an electron's orbital velocity to the speed of light ($v/c$), bridging the gap between classical orbits and relativistic quantum mechanics. 

In this model, we evolve this geometric tradition by interpreting $\alpha$ as the \textbf{scaling limit of information erasure} within de Gosson's finite-support phase space. Here, $\alpha$ represents the density of addressable macrostates relative to the total available action:

\begin{itemize}
    \item \textbf{The Erasure Ratio ($\alpha_1$):} Redefines the classical electron radius $r_e$ as the \textit{erasure distance} $\delta_q$ required to map a microstate to a binary address relative to the cell diameter.
    \item \textbf{The Hamiltonian Ratio ($\alpha_2$):} Defines the Bohr radius $a_0$ as the \textit{system support} boundary $\Delta q$, encompassing exactly $137$ units of action.
    \item \textbf{The Synthesis:} Just as Sommerfeld linked the velocity scales of the atom, this model links the \textbf{thermodynamic erasure scale} to the \textbf{geometric action scale}.
\end{itemize}

\begin{equation}
\label{eq:SommerfeldInfo}
\alpha = \underbrace{\frac{v}{c}}_{\text{Sommerfeld}} \implies \underbrace{\frac{\delta_q}{\delta_{max}}}_{\text{Erasure Density}}
\end{equation}

This shift suggests that $\alpha$ is a fundamental constant of \textbf{computational capacity} in phase space, determining how much information can be "squeezed" into the hard boundaries of a de Gosson quantum blob before the continuum collapses into discrete observables.

\end{document}
