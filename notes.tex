\documentclass[11pt]{article}

% --- Essential Packages ---
\usepackage[utf8]{inputenc}
\usepackage[margin=1in]{geometry}
\usepackage{amsmath, amssymb} % For advanced math and subequations
\usepackage{xcolor}           % For colored boxes in lab notes
\usepackage{hyperref}         % For clickable references

% --- Page Style ---
\pagestyle{headings}
\title{Notes on Information Erasure and Fine Structure}
\author{Research Draft Summary}
\date{\today}

\begin{document}

\maketitle

\section{Justifying Conjugate Action Quanta}

This section establishes the theoretical framework for partitioning the total action of a classical harmonic oscillator into conjugate components. By integrating symplectic geometry with information theory, we derive the constraints that govern the transition from coarse-grained macrostates to the underlying quantum-statistical density.

\subsection*{The Hamiltonian as a Symmetric Surface}

Rather than beginning with the equations of motion, we define the Hamiltonian $H(q, p)$ as the geometric constraint that defines the system's state in phase space. For the harmonic oscillator, the Hamiltonian is the quadratic sum that represents the conservation of the total action:

\begin{equation}
H(q, p) = \frac{p^2}{2m} + \frac{1}{2}m\omega^2 q^2 = E
\end{equation}

This form is chosen because it treats $q$ and $p$ as the two "coordinates" of an energy-surface. In this view:
\begin{itemize}
    \item \textbf{Geometric Equivalence:} The terms $\frac{p^2}{2m}$ and $\frac{1}{2}m\omega^2 q^2$ are the conjugate energy-components. At the scales $\Delta q$ and $\Delta p$, these components reach their maximums, such that $E = \frac{(\Delta p/2)^2}{2m} = \frac{1}{2}m\omega^2 (\Delta q/2)^2$.
    \item \textbf{The Area-Energy Link:} This quadratic constraint is what forces the phase space trajectory into an ellipse. The total area $A$ enclosed by this energy-surface is directly proportional to the energy: $A = \frac{2\pi E}{\omega}$.
\end{itemize}


\subsection*{Energy Balance and the Symmetric Differential Form}

Before arriving at the principle of stationary action, we observe a natural symmetry in the energy dynamics. For a total energy $H(q, p) = E$, the differential energy identity along a trajectory is:
\begin{equation}
dH = \frac{\partial H}{\partial q} dq + \frac{\partial H}{\partial p} dp = (m\omega^2 q) \, dq + \left(\frac{p}{m}\right) dp = 0
\end{equation}

Using Hamilton's equations, we examine the differential contributions of $p \, dq$ and $q \, dp$:
\begin{itemize}
    \item \textbf{Kinetic Displacement:} $p \, dq$ represents infinitesimal work over a coordinate shift.
    \item \textbf{Potential Displacement:} $q \, dp$ represents the conjugate momentum shift over the coordinate's extent.
\end{itemize}
This motivates the symmetric term $\frac{1}{2}(p \, dq - q \, dp)$ as the natural geometric integrand, encapsulating the "circulation" in $(q, p)$ phase space.

\subsection*{Conjugate Action Quanta and Symplectic Symmetry}

The symplectic two-form $\omega = dq \wedge dp$ implies the area element is composed of the $(q, p)$ pairing. We define a pair of \textbf{conjugate action quanta}, $A_q$ and $A_p$:

\textbf{Geometric Motivation:}
The phase space trajectory forms an ellipse with total area $A$. We express this area using the symmetric relationship between the conjugate variables:
\begin{equation}
A = \oint p \, dq = \left| \oint q \, dp \right| = \frac{1}{2} \oint (p \, dq - q \, dp) = \frac{\pi}{4} \Delta q \Delta p
\end{equation}

\textbf{Partitioning and Equipartition:}
In the semiclassical ground state where $A = \pi \hbar$:
\begin{equation}
A = A_q + A_p = \pi \hbar \implies A_q = A_p = \frac{\pi \hbar}{2}
\end{equation}
This decomposition suggests each canonical dimension contributes exactly half of the total action quantum, aligning with the symplectic invariance of the system.

\subsection*{Information-Theoretic Resolution and Landauer Erasure}

The transition from microstates to macrostates is driven by information content. We consider $\mu_q, \mu_p$ as indistinguishable microstates within the action quanta, while $q, p$ are the coarse-grained macrostates emergent after \textbf{Landauer erasure}.

\textbf{Primary Information-Energy Constraint:}
The complexity is governed by the bit-count $b$ of the binary strings $s^*_q, s^*_p$:
\begin{equation}
b E \tau \le \pi \hbar, \quad b = |s^*_q| + |s^*_p|
\end{equation}

\textbf{Local Deviations and Action Bounds:}
Defining the deviations $\delta_q = \mu_q - q$ and $\delta_p = \mu_p - p$, we bound the local action variations:
\begin{equation}
A_q = \frac{\pi}{4} |\delta_q| \Delta p \le \frac{\pi \hbar}{2}, \quad A_p = \frac{\pi}{4} \Delta q |\delta_p| \le \frac{\pi \hbar}{2}
\end{equation}

\textbf{The Unified Identity:}
\begin{equation}
b E \tau \le A_q + A_p \le \frac{\pi}{4} \Delta q \Delta p \le \pi \hbar
\end{equation}

\textbf{Emergent Density:}
As system energy $E \to \infty$, the density of $\delta_q$ over $q$ and $\delta_p$ over $p$ converges to the probability densities of the QHO. The coarse-graining "smears" the microstate into the macrostate according to these informational limits.

\section{The Information-Action Limit}
The available action for erasure inside de Gosson's action quanta is defined by:
\begin{equation}
\label{eq:InfoActionLimit}
b E \tau \le \pi \hbar 
\end{equation}
where $b$ is the bit count, $E$ is Landauer erasure energy, and $\pi \hbar$ represents the minimal elliptical phase-space area.

\section{Erasure in Conjugate Action Quanta}
Erasure within a conjugate pair of Heisenberg cells is bounded as follows:
\begin{subequations}
\label{eq:SymplecticBounds}
\begin{equation}
\label{eq:SymplecticBoundsq}
\frac{\pi}{4}|\mu_q - q|\Delta p \le \frac{\pi \hbar}{2} 
\end{equation}
\begin{equation}
\label{eq:SymplecticBoundsp}
\frac{\pi}{4}\Delta q|\mu_p - p| \le \frac{\pi \hbar}{2} 
\end{equation}
\end{subequations}
Each coordinate displacement $(\mu_q - q)$ utilizes exactly half of the available de Gosson action.

\section{Fine Structure Scaling}
The fine structure constant $\alpha$ emerges from the geometric scaling of these boundaries:
\begin{equation}
\alpha = \frac{|\delta_q| \Delta p}{2 \hbar} = \frac{2 \hbar}{\Delta q \Delta p}
\end{equation}
The global scaling between the erasure shift and the system boundary follows the squared relation:
\begin{equation}
\alpha^2 = \frac{\delta_q}{\Delta q}
\end{equation}

\vspace{2em}

% --- Lab Notebook Summary Box ---
\begin{center}
\fcolorbox{black}{gray!10}{\begin{minipage}{0.9\textwidth}
\textbf{Lab Note: Geometric Derivation of $\alpha$ via de Gosson Quanta} \\
\textit{Core Concept:} $\alpha$ scales the Landauer erasure distance to the hard Heisenberg boundaries of the de Gosson "quantum blob."

\begin{itemize}
    \item \textbf{Action Partition:} Each erasure utilizes exactly $\pi\hbar/2$ of the conjugate pair. 
    \item \textbf{Hamiltonian Scaling:} For $\alpha \approx 1/137$, the total support $\Delta q \Delta p$ encompasses $\alpha^{-1}$ units of the action bound ($2\hbar$). 
    \item \textbf{Mappings:} $\delta_q \to r_e$ (Classical Radius), $\Delta q \to a_0$ (Bohr Radius).
    \item \textbf{Information Link:} Bit complexity $b \propto \alpha^{-1}$, suggesting stability is reached at the Landauer limit across exactly 137 action quanta.
\end{itemize}
\end{minipage}}
\end{center}


\subsection*{Historical Context: From Sommerfeld to Information Erasure}

Originally introduced by \textbf{Arnold Sommerfeld} in 1916, the fine-structure constant $\alpha \approx 1/137$ was defined as the ratio of an electron's orbital velocity to the speed of light ($v/c$), bridging the gap between classical orbits and relativistic quantum mechanics. 

In this model, we evolve this geometric tradition by interpreting $\alpha$ as the \textbf{scaling limit of information erasure} within de Gosson's finite-support phase space. Here, $\alpha$ represents the density of addressable macrostates relative to the total available action:

\begin{itemize}
    \item \textbf{The Erasure Ratio ($\alpha_1$):} Redefines the classical electron radius $r_e$ as the \textit{erasure distance} $\delta_q$ required to map a microstate to a binary address relative to the cell diameter.
    \item \textbf{The Hamiltonian Ratio ($\alpha_2$):} Defines the Bohr radius $a_0$ as the \textit{system support} boundary $\Delta q$, encompassing exactly $137$ units of action.
    \item \textbf{The Synthesis:} Just as Sommerfeld linked the velocity scales of the atom, this model links the \textbf{thermodynamic erasure scale} to the \textbf{geometric action scale}.
\end{itemize}

\begin{equation}
\label{eq:SommerfeldInfo}
\alpha = \underbrace{\frac{v}{c}}_{\text{Sommerfeld}} \implies \underbrace{\frac{\delta_q}{\delta_{max}}}_{\text{Erasure Density}}
\end{equation}

This shift suggests that $\alpha$ is a fundamental constant of \textbf{computational capacity} in phase space, determining how much information can be "squeezed" into the hard boundaries of a de Gosson quantum blob before the continuum collapses into discrete observables.

\end{document}
