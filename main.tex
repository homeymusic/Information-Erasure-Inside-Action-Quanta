\documentclass[%
 aps, reprint, prl, amsmath,amssymb
]{revtex4-2}

\usepackage{graphicx}% Include figure files
\graphicspath{{figures/}} 
\usepackage{dcolumn}% Align table columns on decimal point
\usepackage{bm}% bold math
\usepackage{orcidlink}
\usepackage{mathtools}

% -------------- Watermark ----------------
\usepackage[style=iso]{datetime2}
\usepackage{FiraMono} 
\usepackage{tikz}
\usepackage{transparent} 

% Use shipout/background to ensure it stays behind text/figures
\AddToHook{shipout/background}{
  \begin{tikzpicture}[remember picture, overlay]
    \node [
      rotate=55,
      scale=1.2,
      text opacity=0.15,      
      color=gray!50,          
      font=\fontfamily{FiraMono-TLF}\selectfont\bfseries,
      align=center
    ] at (current page.center) {
        {\fontsize{60}{70}\selectfont WORKING DRAFT} \\ [0.8cm]
        {\Huge \DTMnow} \\
        {\Huge doi.org/10.17605/OSF.IO/EV8H6} \\ [0.8cm]
        {\Large Computational Complex Systems Laboratory} \\
        {\Large Homey Music, Detroit, USA} \\ [0.2cm]
         {\transparent{0.25}\includegraphics[width=1.0cm]{logo.png}}
    };
  \end{tikzpicture}
}
% -----------------------------------------


\DeclareMathOperator*{\argmin}{argmin}
\begin{document}

\title{Information Erasure Inside Action Quanta}

\begin{abstract}
Quantum microstates are operationally indistinguishable within finite-support Heisenberg cells. Inside conjugate pairs of these de Gosson action quanta, governed by the principle of least action, Landauer information erasure precludes the classical continuum, necessitating discrete configurations and the emergence of quantum mechanics.
\end{abstract}

\author{Brian S. Mulloy\orcidlink{0000-0002-1803-3172}}
\email{brian@homeymusic.com}
\affiliation{Computational Complex Systems Laboratory\\Homey Music, Corktown, Detroit, MI, USA}
\date{\today}

\maketitle

\section{Figures}

\begin{figure}[t!]
    \centering
    \includegraphics[width=\linewidth]{erase.pdf}
\caption{\textbf{Information erasure inside conjugate action quanta} (schematic). 
System boundaries $\Delta q$ and $\Delta p$ determine the total action $A = \pi \hbar$, while the reciprocal relations $\delta_q = 2\hbar / \Delta p$ and $\delta_p = 2\hbar / \Delta q$ characterize the symplectic squeezing of the conjugate pair of action quanta: $A_q = \frac{\pi}{4} \delta_q \Delta p = \frac{\pi \hbar}{2}$ and $A_p = \frac{\pi}{4} \Delta q \delta_p = \frac{\pi \hbar}{2}$. The variables $\epsilon_q$ and $\epsilon_p$ represent the erasure distances from the microstate $\mu_{q,p}$ with binary encoding $s_\mu$ to the macrostate $(q,p)$ with the minimal binary encoding $\mathbf{s^*}$, where bit-length $|\mathbf{s^*}| \le |s_\mu|$; the physical scales satisfy $A = A_q + A_p = \pi \hbar$, $\epsilon_q \le \delta_q \le \Delta q$, and $\epsilon_p \le \delta_p \le \Delta p$.}

\label{fig:erase}
\end{figure}

\section{Equations}

\subsection{The Information-Action Limit}

The action available for information erasure inside a conjugate pair of finite-support Heisenberg cells, de Gosson action quanta, is
\begin{equation}
\label{eq:InfoActionLimit}
b E \tau \le \pi \hbar ,
\end{equation}
where $b = b_q + b_p = |\mathbf{s}^*_q| + |\mathbf{s}^*_p|$ is the count of bits in the minimal binary encoding $\mathbf{s}^*_q, \mathbf{s}^*_p$ of the physical state $(q,p)$, $E = k_B T \ln 2$ is the Landauer erasure energy, and $\tau$ is the erasure time. Here $\pi \hbar$ represents a conjugate pair of de Gosson action quanta, defined as the minimal phase-space area of finite support; unlike statistical variance $\sigma_q \sigma_p$, the boundaries $\Delta q$ and $\Delta p$ define the hard Heisenberg limits where states are operationally indistinguishable \cite{DeGosson2003}.






\section{Information Erasure Inside the Conjugate Action Quanta}

\subsection{The General Form: Conjugate Action Quanta}

The physical action available for information erasure is constrained within a \textbf{conjugate pair} of de Gosson action quanta. The total action, $A_q + A_p = \pi \hbar$, partitions between the spatial and momentum quanta according to the relative geometric proportions of the phase-space supports, $\Delta q$ and $\Delta p$:
\begin{equation}
\label{eq:GeneralPartition}
A_q = \frac{\pi \hbar}{1 + \frac{\Delta p}{\Delta q}}, \quad A_p = \frac{\pi \hbar}{1 + \frac{\Delta q}{\Delta p}}
\end{equation}
These quanta are ellipses with areas defined by their conjugate supports $A_q = \frac{\pi}{4}\delta_q\Delta p$ and $A_p = \frac{\pi}{4}\Delta q \delta_p$. The squeezed dimensions of each quanta, $\delta_q$ and $\delta_p$, represent the maximum deterministic erasure defined by the hard \textbf{Heisenberg limits} where microstates are \textbf{operationally indistinguishable}. These capacities emerge as the symplectic projections of the action:
\begin{equation}
\label{eq:GeneralCapacities}
\delta_q = \frac{4 A_q}{\pi \Delta p}, \quad \delta_p = \frac{4 A_p}{\pi \Delta q}
\end{equation}
The conjugate cells are tiled across phase-space. The discrete counts $N_q$ and $N_p$ are dictated by the geometric proportions of the conjugate supports, ensuring a minimum of a single fundamental cell through a floor function constraint:
\begin{equation}
\label{eq:GeneralTiling}
N_q = \max\left(1, \left\lfloor \frac{\Delta p}{\Delta q} \right\rfloor\right), \quad N_p = \max\left(1, \left\lfloor \frac{\Delta q}{\Delta p} \right\rfloor\right)
\end{equation}
The structure of the state emerges as microstates with excess action $b E \tau$ are irreversibly erased to macrostates possessing only the action supported by these discrete cells.

\subsection{The Harmonic Oscillator: Symmetrical Quanta}

The Harmonic Oscillator represents a state of dynamical equilibrium. In the ground state, the potential and kinetic energies are balanced, resulting in symmetric supports where $\Delta q = \Delta p$.

\textbf{Explicit Substitution for Action:}
The symmetry of the supports ($\Delta q = \Delta p$) into Eq. (\ref{eq:GeneralPartition}) yields:
\begin{equation}
A_q = \frac{\pi \hbar}{1 + 1} = \frac{\pi \hbar}{2}, \quad A_p = \frac{\pi \hbar}{1 + 1} = \frac{\pi \hbar}{2}
\end{equation}

\textbf{Explicit Substitution for Capacity:}
The resulting capacities emerge from Eq. (\ref{eq:GeneralCapacities}) as:
\begin{equation}
\delta_q = \frac{4 (\pi \hbar / 2)}{\pi \Delta p} = \frac{2 \hbar}{\Delta p}, \quad \delta_p = \frac{4 (\pi \hbar / 2)}{\pi \Delta q} = \frac{2 \hbar}{\Delta q}
\end{equation}

\textbf{Explicit Substitution for Tiling:}
The symmetric supports in Eq. (\ref{eq:GeneralTiling}) yield the unitary tiling of the ground state:
\begin{equation}
N_q = \max(1, \lfloor 1 \rfloor) = 1, \quad N_p = \max(1, \lfloor 1 \rfloor) = 1
\end{equation}

\subsection{The Particle in a Box: Asymmetrical Quanta}

In a box of width $L$, the spatial support is fixed ($\Delta q = L$). This physical constraint forces the \textbf{conjugate pair} into a geometric asymmetry as the momentum support $\Delta p$ increases.

\textbf{Explicit Substitution for Action:}
Substituting the fixed boundary $\Delta q = L$ into Eq. (\ref{eq:GeneralPartition}):
\begin{equation}
A_q = \frac{\pi \hbar}{1 + \frac{\Delta p}{L}}, \quad A_p = \frac{\pi \hbar}{1 + \frac{L}{\Delta p}}
\end{equation}

\textbf{Explicit Substitution for Capacity:}
The capacities reflect the divergent scaling of the supports:
\begin{equation}
\delta_q = \frac{4 \hbar L}{\Delta p (L + \Delta p)}, \quad \delta_p = \frac{4 \hbar \Delta p}{L (\Delta p + L)}
\end{equation}

\textbf{Explicit Substitution for Tiling ($N_q$ and $N_p$):}
The fixed spatial boundary $L$ in Eq. (\ref{eq:GeneralTiling}) dictates the emergent resolution:
\begin{equation}
N_p = \max\left(1, \left\lfloor \frac{L}{\Delta p} \right\rfloor\right) = 1, \quad N_q = \max\left(1, \left\lfloor \frac{\Delta p}{L} \right\rfloor\right)
\end{equation}

\textbf{The Asymmetric Result:}
As $\Delta p$ increases, the momentum resolution remains at its fundamental limit ($N_p = 1$), indicating that all momentum microstates within the support are \textbf{operationally indistinguishable}. Simultaneously, $N_q$ grows with the momentum support. The spatial support $L$ is resolved with a density of $N_q$ tiles.
f
\subsection{The Rigid Rotor: Dynamic Quanta}

The third iconic pillar of the model is the \textbf{Rigid Rotor}. While the Harmonic Oscillator represents vibration and the Particle in a Box represents translation, the Rigid Rotor introduces pure angular motion. In phase space, this motion is characterized by the rotation of the supports $\Delta q(t)$ and $\Delta p(t)$ around the origin. Let $\theta(t)$ represent the instantaneous phase angle, where the geometric aspect ratio is defined by the trigonometric relationship $\kappa(\theta) = \tan \theta$.

\textbf{Explicit Substitution for Action:}
Substituting the rotational aspect ratio $\kappa(\theta)$ into Eq. (\ref{eq:GeneralPartition}) demonstrates how the action quanta precess between dimensions during a single period:
\begin{equation}
A_q(\theta) = \frac{\pi \hbar}{1 + \tan \theta}, \quad A_p(\theta) = \frac{\pi \hbar}{1 + \cot \theta}
\end{equation}
This confirms that the "erasure budget" is a dynamic quantity that follows the rotational orientation of the state.

\textbf{Explicit Substitution for Capacity:}
The resulting capacities $\delta_q$ and $\delta_p$ track the angular position of the rotor, revealing the breathing of the Heisenberg limits as the state sweeps through phase space:
\begin{equation}
\delta_q(\theta) = \frac{4 \hbar \cos \theta}{\Delta p (\cos \theta + \sin \theta)}, \quad \delta_p(\theta) = \frac{4 \hbar \sin \theta}{\Delta q (\sin \theta + \cos \theta)}
\end{equation}

\textbf{Explicit Substitution for Tiling ($N_q$ and $N_p$):}
The discrete resolution of the state fluctuates as the rotor cycles, governed by the angular tiling counts:
\begin{equation}
N_q(\theta) = \max\left(1, \left\lfloor \tan \theta \right\rfloor\right), \quad N_p(\theta) = \max\left(1, \left\lfloor \cot \theta \right\rfloor\right)
\end{equation}

\textbf{The Dynamic Result:}
The Rigid Rotor reveals that \textbf{information erasure} is a phase-dependent conservation law. As the system rotates, microstate detail is periodically erased in one dimension to be resolved in the conjugate dimension. This ensures that the symplectic area $\pi \hbar$ is preserved as a covariant property, while the operational indistinguishability of the state oscillates in dynamic equilibrium.






\section{Thermodynamics and Geometry}

These equations relate by

\begin{equation}
\label{eq:ThermoGeo}
A_L \le A \le A_{d\Gamma} = \pi \hbar,
\end{equation}

where $A_L \coloneqq b E \tau$, $A = A_q + A_p$, and $A_{d\Gamma} \coloneqq \pi \hbar$. 

\subsection{Fine Structure}

There is a self-similar hierarchy $|\epsilon_q| \le \delta_q \le \Delta q$ where $\alpha$ is the  ratio between three nested physical scales:
\begin{itemize}
        \item \textbf{Erasure $|\epsilon_q|$:} The ``micro'' range where microstates $\mu_q$ are collapsed.
        \item \textbf{Squeezed $\delta_q$:} The ``meso'' range of the squeezed Heisenberg cell boundary.
        \item \textbf{Quantum $\Delta q$:} The ``macro'' range of the Hamiltonian support.
    \end{itemize}
    \textit{Insight:} Each level of the hierarchy ``erases'' the same proportion of information ($\alpha$) to move to the next level of coarse-graining.

\begin{equation}
\label{eq:alpha1}
\alpha_\epsilon = \frac{|\epsilon_q|}{\delta_q} = \frac{|\epsilon_q| \Delta p}{2 \hbar}
\end{equation} where $\epsilon_q \coloneqq \mu_q - q$ is the erasure distance and $\delta_q \coloneqq\frac{2 \hbar}{\Delta p}$ is the quantum of action coordinate bound for $A_q$. The ratio of the coordinate quantum of action bound to the overall Hamiltonian (akin to $\lambda_e/a_0$) is

\begin{equation}
\label{eq:alpha2}
\alpha_\delta =  \frac{\delta_q}{\Delta q} = \frac{2 \hbar}{\Delta q \Delta p}.
\end{equation} And the ratio of the erasure distance to the overall Hamiltonian (akin to $r_e/a_0$)

\begin{equation}
\label{eq:alpha3}
\alpha_\Delta^2 = \frac{|\epsilon_q|}{\Delta q}.
\end{equation}

All together we have \begin{equation}
\label{eq:alpha_combined}
|\epsilon_q| = \alpha \delta_q = \alpha^2 \Delta q
\end{equation}

\section{Normalized Equations}

\subsection{Information Erasure Inside Action Quanta}

Landauer erasure within a conjugate pair of Heisenberg cells is bounded by:
\begin{equation}
\label{eq:SymplecticBounds}
|q_? - \mathbf{Q}^*| \le \mathcal{P}^{-1}, 
\quad 
|p_? - \mathbf{P}^*| \le \mathcal{Q}^{-1},
\end{equation}
where the operationally indistinguishable microstates are represented as $\mu_q \coloneqq q_0 q_?$ and $\mu_p \coloneqq p_0 p_?$, while the macrostates $\mathbf{Q}^*$ and $\mathbf{P}^*$, encoded with the action available in each orthogonal quanta, are defined as $q \coloneqq q_0 \mathbf{Q}^*$ and $p \coloneqq p_0 \mathbf{P}^*$. The maximal extent and maximal momentum of the system are represented as $\Delta q \coloneqq q_0 \mathcal{Q}$ and $\Delta p \coloneqq p_0 \mathcal{P}$ 

\paragraph{Referencing Example:}
The symplectic squeezing is captured by Eq.~(\ref{eq:SymplecticBounds}).

\section{Foundational Manuscripts}

\begin{itemize}
    \item \textbf{Geometry}
    \begin{itemize}
        \item Heisenberg \cite{Heisenberg1927}
        \item de Gosson \cite{DeGosson2003}
    \end{itemize}
    
    \item \textbf{Thermodynamics}
    \begin{itemize}
        \item Landauer \cite{Landauer1961}
        \item Plenio and Vitelli \cite{PlenioVitelli2001}
        \item Margolus and Levitin \cite{MargolusLevitin1998}
    \end{itemize}
    
    \item \textbf{Methodology}
    \begin{itemize}
        \item Stern \cite{Stern1858}
        \item Brocot \cite{Brocot1861}
        \item Graham, Knuth, and Patashnik \cite{ConcreteMath}
        \item Stolzenburg \cite{Stolzenburg2015}
        \item Aiylam \cite{Aiylam2013}
    \end{itemize}
\end{itemize}

\bibliography{main} 
\end{document}
