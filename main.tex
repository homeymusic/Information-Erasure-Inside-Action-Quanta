\documentclass[%
 aps, reprint, prl, amsmath,amssymb
]{revtex4-2}

\usepackage{graphicx}% Include figure files
\graphicspath{{figures/}} 
\usepackage{dcolumn}% Align table columns on decimal point
\usepackage{bm}% bold math
\usepackage{orcidlink}
\usepackage{mathtools}
\usepackage{algorithmic}
\usepackage{algpseudocode}

% -------------- Watermark ----------------
\usepackage[style=iso]{datetime2}
\usepackage{FiraMono} 
\usepackage{tikz}
\usepackage{transparent} 

% Use shipout/background to ensure it stays behind text/figures
\AddToHook{shipout/background}{
  \begin{tikzpicture}[remember picture, overlay]
    \node [
      rotate=55,
      scale=1.2,
      text opacity=0.15,      
      color=gray!50,          
      font=\fontfamily{FiraMono-TLF}\selectfont\bfseries,
      align=center
    ] at (current page.center) {
        {\fontsize{60}{70}\selectfont WORKING DRAFT} \\ [0.8cm]
        {\Huge \DTMnow} \\
        {\Huge doi.org/10.17605/OSF.IO/EV8H6} \\ [0.8cm]
        {\Large Computational Complex Systems Laboratory} \\
        {\Large Homey Music, Detroit, USA} \\ [0.2cm]
         {\transparent{0.25}\includegraphics[width=1.0cm]{logo.png}}
    };
  \end{tikzpicture}
}
% -----------------------------------------


\DeclareMathOperator*{\argmin}{argmin}
\begin{document}

\title{Information Erasure Inside Action Quanta}

\begin{abstract}
Quantum microstates are operationally indistinguishable within finite-support Heisenberg cells. Inside conjugate pairs of these de Gosson action quanta, governed by the principle of least action, Landauer information erasure precludes the classical continuum, necessitating discrete configurations and the emergence of quantum mechanics.
\end{abstract}

\author{Brian S. Mulloy\orcidlink{0000-0002-1803-3172}}
\email{brian@homeymusic.com}
\affiliation{Computational Complex Systems Laboratory\\Homey Music, Corktown, Detroit, MI, USA}
\date{\today}

\maketitle

\section{Figures}

\begin{figure}[t!]
    \centering
    \includegraphics[width=\linewidth]{erase.pdf}
\caption{\textbf{Information erasure inside conjugate action quanta} (schematic). 
System boundaries $\Delta q$ and $\Delta p$ determine the total action $A = \pi \hbar$, while the reciprocal relations $\delta_q = 2\hbar / \Delta p$ and $\delta_p = 2\hbar / \Delta q$ characterize the symplectic squeezing of the conjugate pair of action quanta: $A_q = \frac{\pi}{4} \delta_q \Delta p = \frac{\pi \hbar}{2}$ and $A_p = \frac{\pi}{4} \Delta q \delta_p = \frac{\pi \hbar}{2}$. The variables $\epsilon_q$ and $\epsilon_p$ represent the erasure distances from the microstate $\mu_{q,p}$ with binary encoding $s_\mu$ to the macrostate $(q,p)$ with the minimal binary encoding $\mathbf{s^*}$, where bit-length $|\mathbf{s^*}| \le |s_\mu|$; the physical scales satisfy $A = A_q + A_p = \pi \hbar$, $\epsilon_q \le \delta_q \le \Delta q$, and $\epsilon_p \le \delta_p \le \Delta p$.}

\label{fig:erase}
\end{figure}









\section{Equations}

\subsection{The Information-Action Limit}

The action available for information erasure within a conjugate pair of finite-support Heisenberg cells, the action quanta, is
\begin{equation}
\label{eq:InfoActionLimit}
b E \tau \le \pi \hbar ,
\end{equation}
where $b = b_q + b_p = |\mathbf{s}^*_q| + |\mathbf{s}^*_p|$ is the count of bits in the minimal binary encoding $\mathbf{s}^*_q, \mathbf{s}^*_p$ of the physical state $(q,p)$, $E = k_B T \ln 2$ is the Landauer erasure energy, and $\tau$ is the erasure time. The total action $\pi \hbar$ is the minimal phase-space area of finite support; unlike statistical variance $\sigma_q \sigma_p$, the boundaries $\Delta q$ and $\Delta p$ define the hard Heisenberg limits where states are operationally indistinguishable \cite{DeGosson2003}.

\section{Information Erasure Inside the Conjugate Action Quanta}

\subsection{The General Form: Conjugate Action Quanta}

The physical action available for information erasure is constrained within a conjugate pair of action quanta. The total action, $A_q + A_p = \pi \hbar$, partitions between the spatial and momentum quanta according to the relative geometric proportions of the phase-space supports, $\Delta q$ and $\Delta p$.

\textit{Action Partition:}
\begin{equation}
\label{eq:GeneralPartition}
A_q = \frac{\pi \hbar}{1 + \left ( \frac{\Delta p / p_0}{\Delta q / q_0}\right )^2}, \quad A_p = \frac{\pi \hbar}{1 + \left (\frac{\Delta q/q_0}{\Delta p/p_0}\right )^2 }
\end{equation}

\textit{Symplectic Capacities:}
\begin{equation}
\label{eq:GeneralCapacities}
\delta_q = \frac{4 A_q}{\pi \Delta p}, \quad \delta_p = \frac{4 A_p}{\pi \Delta q}
\end{equation}

\textit{Discrete Tiling:}
\begin{equation}
\label{eq:GeneralTiling}
N_q = \max\left(1, \left\lfloor \frac{\Delta p/p_0}{\Delta q / q_0} \right\rfloor\right), \quad N_p = \max\left(1, \left\lfloor \frac{\Delta q/q_0}{\Delta p/p_0} \right\rfloor\right)
\end{equation}

The supports $\Delta p$ and $\Delta q$ define the macrostate boundaries, while the minor diameters $\delta_q$ and $\delta_p$ define the resolution limit—the threshold of the hard Heisenberg limits where microstates become operationally indistinguishable.

\subsection{The Harmonic Oscillator: Symmetrical Quanta}

The Harmonic Oscillator is a state of dynamical equilibrium. In the ground state, the potential and kinetic energies are balanced, resulting in symmetric supports where $\Delta q = \Delta p$.

\textit{Action Partition:}
\begin{equation}
A_q = \frac{\pi \hbar}{2}, \quad A_p = \frac{\pi \hbar}{2}
\end{equation}

\textit{Symplectic Capacities:}
\begin{equation}
\delta_q = \frac{2 \hbar}{\Delta p}, \quad \delta_p = \frac{2 \hbar}{\Delta q}
\end{equation}

\textit{Discrete Tiling:}
\begin{equation}
N_q = 1, \quad N_p = 1
\end{equation}

\subsection{The Particle in a Box: Asymmetrical Quanta}

In a box of width $L$, the spatial support is fixed ($\Delta q = L$). This physical constraint forces the conjugate pair into a geometric asymmetry as the momentum support $\Delta p$ increases relative to the spatial scale $q_0$.

\textit{Action Partition:}
\begin{equation}
A_q = \frac{\pi \hbar}{1 + \left ( \frac{\Delta p/p_0}{L/q_0}\right )^2}, \quad A_p = \frac{\pi \hbar}{1 + \left (\frac{L/q_0}{\Delta p / p_0} \right )^2}
\end{equation}

\textit{Symplectic Capacities:}
\begin{equation} 
\delta_q = \frac{4 \hbar}{\Delta p \left( 1 + \left[ \frac{\Delta p/p_0}{L/q_0} \right]^2 \right)}, \quad \delta_p = \frac{4 \hbar}{L \left( 1 + \left[ \frac{L/q_0}{\Delta p/p_0} \right]^2 \right)} 
\end{equation}

\textit{Discrete Tiling:}
\begin{equation}
N_q = \max\left(1, \left\lfloor \frac{\Delta p/p_0}{L/q_0} \right\rfloor\right), \quad N_p = 1
\end{equation}

As $\Delta p$ increases, the momentum resolution remains at its fundamental limit, while the spatial support $L$ is resolved by an increasing count of tiles defined by the squeezed spatial capacity $\delta_q$.

\subsection{The Free Particle: Conjugate Action Redistribution}

In the Free Wave Packet, the momentum support remains a constant of motion $p_0$, while the spatial support $\Delta q(t)$ increases linearly. This expansion is a continuous redistribution of action within the conjugate pair. 

\textit{Action Partition:}
\begin{equation}
A_q(t) = \frac{\pi \hbar}{1 + \left ( \frac{1}{\Delta q(t) / q_0}\right )^2}, \quad A_p(t) = \frac{\pi \hbar}{1 + \left ( \Delta q(t)/q_0 \right )^2 }
\end{equation}

\textit{Symplectic Capacities:}
\begin{equation}
\delta_q(t) = \frac{4 A_q(t)}{\pi p_0}, \quad \delta_p(t) = \frac{4 A_p(t)}{\pi \Delta q(t)}
\end{equation}

\textit{Discrete Tiling:}
\begin{equation}
N_q(t) = \max\left(1, \left\lfloor \frac{1}{\Delta q(t) / q_0} \right\rfloor\right), \quad N_p(t) = \max\left(1, \left\lfloor \Delta q(t) / q_0 \right\rfloor\right)
\end{equation}

The total action budget is conserved as the spatial dimension draws action from its conjugate momentum partner, allowing the state to maintain its operational structure as it spreads.




\subsection{The Rotational State: Phase-Dependent Quanta}

In systems with periodic symmetry, the phase angle $\phi$ dictates the geometric proportions of the phase-space supports, $\Delta q = q_0 |\cos \phi|$ and $\Delta p = p_0 |\sin \phi|$. The resulting aspect ratio acts as the pivot of the information budget: 
\begin{equation}
\tan \phi = \frac{\Delta p / p_0}{\Delta q / q_0}
\end{equation}

\textit{Action Partition:}
The total action $\pi \hbar$ is partitioned into spatial and momentum components according to this orientation:
\begin{equation}
A_q = \pi \hbar \cos^2 \phi, \quad A_p = \pi \hbar \sin^2 \phi
\end{equation}

\textit{Symplectic Capacities (The Resolution):}
The resolution limit $\delta$ is the result of dividing the allocated action by the available conjugate window. Using the geometric scaling factor $4/\pi$ for the phase-space cell, the derivation for the spatial resolution is:
\begin{equation}
\delta_q = \frac{4 A_q}{\pi \Delta p} = \frac{4 (\pi \hbar \cos^2 \phi)}{\pi (p_0 |\sin \phi|)} \implies \delta_q = \frac{4 \hbar \cos^2 \phi}{p_0 |\sin \phi|}
\end{equation}
Similarly, the momentum resolution is constrained by the spatial support:
\begin{equation}
\delta_p = \frac{4 A_p}{\pi \Delta q} = \frac{4 (\pi \hbar \sin^2 \phi)}{\pi (q_0 |\cos \phi|)} \implies \delta_p = \frac{4 \hbar \sin^2 \phi}{q_0 |\cos \phi|}
\end{equation}

\textit{Discrete Tiling:}
The number of addressable states, or "bins," available in each dimension is given by the floor of the support ratios:
\begin{equation}
N_q = \max(1, \lfloor |\tan \phi| \rfloor), \quad N_p = \max(1, \lfloor |\cot \phi| \rfloor)
\end{equation}

\subsection{The Rigid Rotor: Dynamic Action Partition}

The Rigid Rotor is a system in continuous rotation where the phase angle evolves as $\phi(t) = \omega t$. This maps the physical state to the rotation of the conjugate action quanta in phase space. The reciprocal ellipses are driven by the phase to maintain the information-action limit against a fixed radial support $R$ and the conjugate momentum scale $p_{\omega} = \hbar/R$.

\textit{Action Partition:}
\begin{equation}
A_q(t) = \pi \hbar \cos^2(\omega t), \quad A_p(t) = \pi \hbar \sin^2(\omega t)
\end{equation}

\textit{Symplectic Capacities:}
\begin{equation}
\delta_q(t) = \frac{4 \hbar \cos^2(\omega t)}{p_{\omega} |\sin(\omega t)|}, \quad \delta_p(t) = \frac{4 \hbar \sin^2(\omega t)}{R |\cos(\omega t)|}
\end{equation}

\textit{Discrete Tiling:}
\begin{equation}
N_q(t) = \max(1, \lfloor |\tan(\omega t)| \rfloor), \quad N_p(t) = \max(1, \lfloor |\cot(\omega t)| \rfloor)
\end{equation}

The breathing of the Heisenberg limits $\delta_q(t)$ and $\delta_p(t)$ ensures that as the system gains resolution in one dimension, it erases it in the other to satisfy the Information-Action Limit.








\subsection{Bell's Theorem: Partitioning Quanta}

In systems involving joint observations at orientations $\alpha$ and $\beta$, the total system budget consists of a pair of conjugate action quanta $\pi \hbar$. Alice and Bob each choose their measurement settings independently; Alice chooses $\alpha$ without knowledge of $\beta$, and Bob chooses $\beta$ without knowledge of $\alpha$. There is no mutual influence, there is only the geometry of the action quanta.

\textit{Action Partition:}
Alice and Bob each perform a local partition of the system's action. Alice’s allocated action $A_\alpha$ is a function strictly of her setting $\alpha$, and Bob’s allocated action $A_\beta$ is a function strictly of his setting $\beta$:
\begin{equation}
A_\alpha = \pi \hbar \cos^2 \alpha, \quad A_\beta = \pi \hbar \sin^2 \beta
\end{equation}

\textit{Symplectic Capacities (Anatomy of Resolution):}
The local resolution limits $\delta_\alpha$ and $\delta_\beta$ are determined by the ratio of the partitioned action to the local conjugate aperture. The observers resolve their portions of the budget using only the local geometry against the characteristic angular scale $\sqrt{\pi \hbar}$:
\begin{equation}
\delta_\alpha = \frac{4 \hbar \cos^2 \alpha}{\sqrt{\pi \hbar} |\sin \alpha|}, \quad \delta_\beta = \frac{4 \hbar \sin^2 \beta}{\sqrt{\pi \hbar} |\cos \beta|}
\end{equation}



\textit{Discrete Tiling:}
The granularity of the local measurement is dictated by the aspect ratio of the local slice. The number of addressable states is a local physical constraint:
\begin{equation}
N_\alpha = \max(1, \lfloor |\tan \alpha| \rfloor), \quad N_\beta = \max(1, \lfloor |\cot \beta| \rfloor)
\end{equation}

\textit{Geometric Proof of Correlation:}
The pattern observed in the aggregate data—typically mapped to the relative phase $\phi = |\alpha - \beta|$—is not an indication of a non-local signal, but is the geometric signature of two independent observers partitioning the same $\pi \hbar$ action budget. 

The observed correlation $E(\alpha, \beta)$ is the difference between the local action partitions. When these independent data sets are compared, the relationship is governed by the relative orientation $\phi$:
\begin{equation}
\cos^2 \alpha - \sin^2 \beta = \cos(\alpha + \beta)\cos(\alpha - \beta)
\end{equation}

Under rotational symmetry in the aggregate data, the $\cos(\alpha + \beta)$ term averages out, leaving the observed expectation value:
\begin{equation}
E(\phi) = -\cos(\alpha - \beta) = -\cos \phi
\end{equation}



\textit{Geometric Conclusion:}
This result is identical to the observed violation of Bell's inequalities, obtained here without non-local influence or hidden variables. The $-\cos \phi$ pattern is the geometric signature of two independent observers partitioning a shared $\pi \hbar$ action budget. The violation of local realism is a retrospective illusion; the resolution limits $\delta_\alpha$ and $\delta_\beta$ are reciprocal because they are local samplings of the same conserved conjugate action quanta. The correlation is not a result of Alice's measurement changing Bob's reality, but is the inevitable geometric alignment of two independent slices of a single, shared physical constant.










\section{Thermodynamics and Geometry}

These equations relate by

\begin{equation}
\label{eq:ThermoGeo}
A_L \le A \le A_{d\Gamma} = \pi \hbar,
\end{equation}

where $A_L \coloneqq b E \tau$, $A = A_q + A_p$, and $A_{d\Gamma} \coloneqq \pi \hbar$. 

\subsection{Fine Structure}

There is a self-similar hierarchy $|\epsilon_q| \le \delta_q \le \Delta q$ where $\alpha$ is the  ratio between three nested physical scales:
\begin{itemize}
        \item \textbf{Erasure $|\epsilon_q|$:} The ``micro'' range where microstates $\mu_q$ are collapsed.
        \item \textbf{Squeezed $\delta_q$:} The ``meso'' range of the squeezed Heisenberg cell boundary.
        \item \textbf{Quantum $\Delta q$:} The ``macro'' range of the Hamiltonian support.
    \end{itemize}
    \textit{Insight:} Each level of the hierarchy ``erases'' the same proportion of information ($\alpha$) to move to the next level of coarse-graining.

\begin{equation}
\label{eq:alpha1}
\alpha_\epsilon = \frac{|\epsilon_q|}{\delta_q} = \frac{|\epsilon_q| \Delta p}{2 \hbar}
\end{equation} where $\epsilon_q \coloneqq \mu_q - q$ is the erasure distance and $\delta_q \coloneqq\frac{2 \hbar}{\Delta p}$ is the quantum of action coordinate bound for $A_q$. The ratio of the coordinate quantum of action bound to the overall Hamiltonian (akin to $\lambda_e/a_0$) is

\begin{equation}
\label{eq:alpha2}
\alpha_\delta =  \frac{\delta_q}{\Delta q} = \frac{2 \hbar}{\Delta q \Delta p}.
\end{equation} And the ratio of the erasure distance to the overall Hamiltonian (akin to $r_e/a_0$)

\begin{equation}
\label{eq:alpha3}
\alpha_\Delta^2 = \frac{|\epsilon_q|}{\Delta q}.
\end{equation}

All together we have \begin{equation}
\label{eq:alpha_combined}
|\epsilon_q| = \alpha \delta_q = \alpha^2 \Delta q
\end{equation}

\section{Normalized Equations}

\subsection{Normalized Harmonic Oscillator}

Landauer erasure for the harmonic oscillator within conjugate action quanta is bounded by:
\begin{equation}
\label{eq:SymplecticBounds}
|q_? - \mathbf{Q}^*| \le \mathcal{P}^{-1}, 
\quad 
|p_? - \mathbf{P}^*| \le \mathcal{Q}^{-1},
\end{equation}
where the operationally indistinguishable microstates are represented as $\mu_q \coloneqq q_0 q_?$ and $\mu_p \coloneqq p_0 p_?$, while the macrostates $\mathbf{Q}^*$ and $\mathbf{P}^*$, encoded with the action available in each orthogonal quanta, are defined as $q \coloneqq q_0 \mathbf{Q}^*$ and $p \coloneqq p_0 \mathbf{P}^*$. The maximal extent and maximal momentum of the system are represented as $\Delta q \coloneqq q_0 \mathcal{Q}$ and $\Delta p \coloneqq p_0 \mathcal{P}$, subject to the support constraints $\mathcal{Q}, \mathcal{P} \ge 1$, which prevent resolution from exceeding the system's unit action scale.

\subsection{Normalized Rigid Rotor}

Landauer erasure for the rigid rotor within conjugate action quanta is bounded by the phase-dependent resolution limits:
\begin{equation}
\label{eq:RotorBounds}
|q_? - \mathbf{Q}^*| \le \frac{4 \cos^2 \phi}{\pi |\sin \phi|}, 
\quad 
|p_? - \mathbf{P}^*| \le \frac{4 \sin^2 \phi}{\pi |\cos \phi|},
\end{equation}
where the operationally indistinguishable microstates are represented as $\mu_q \coloneqq R q_?$ and $\mu_p \coloneqq p_\omega p_?$, while the macrostates $\mathbf{Q}^*$ and $\mathbf{P}^*$, encoded with the action partitioned by the phase angle $\phi$, are defined as $q \coloneqq R \mathbf{Q}^*$ and $p \coloneqq p_\omega \mathbf{P}^*$. 

Physical accessibility is maintained only for non-vanishing conjugate supports; specifically, spatial resolution requires $|\sin \phi| > 0$ while momentum resolution requires $|\cos \phi| > 0$. The divergence of these capacities represents the threshold of total information erasure, mirroring the Heisenberg floor ($\mathcal{P}, \mathcal{Q} \ge 1$) where the system reaches the limit of a single unit action quantum.


\subsection{Normalized Bell's Theorem}

Landauer erasure for the shared action state is bounded by the local resolution limits of Alice and Bob
\begin{equation}
\label{eq:BellBounds}
|\theta_{?\alpha} - \mathbf{\Theta}_\alpha^*| \le \frac{4 \cos^2 \alpha}{\pi |\sin \alpha|}, 
\quad 
|\theta_{?\beta} - \mathbf{\Theta}_\beta^*| \le \frac{4 \sin^2 \beta}{\pi |\cos \beta|},
\end{equation}
where $\theta_{?\alpha}, \theta_{?\beta}$ are microstates erased to macrostates $\mathbf{\Theta}_\alpha^*, \mathbf{\Theta}_\beta^*$. At the unit action scale, the available budget lacks the capacity to encode the precise value of the erasure distances 


\begin{equation}
\label{eq:BellSign}
\epsilon_\alpha = \theta_{?\alpha} - \mathbf{\Theta}_\alpha^*, 
\quad 
\epsilon_\beta = \theta_{?\beta} - \mathbf{\Theta}_\beta^*.
\end{equation}

Consequently, the measurement outcome is truncated to a single bit of information
\begin{equation}
\theta_\alpha = \text{sgn} (\epsilon_{\alpha}), \quad \theta_\beta = \text{sgn} (\epsilon_\beta).
\end{equation}

This binary truncation is a physical requirement of the $\pi \hbar$ system budget. With only two available bits for the joint state, the universe resolves the local sign of the partition while erasing the higher-order geometric details.

Physical accessibility is determined strictly by the local setting; resolution requires non-vanishing conjugate support ($|\sin \alpha| > 0, |\cos \beta| > 0$).




\section{Numerical Simulations}

\begin{algorithmic}
\Function{erase}{$x, \delta_x$}
    \State $\text{left} \gets (-1, 0), \quad \text{right} \gets (1, 0)$
    \State $\text{num} \gets 0, \quad \text{den} \gets 1$
    \State $\mathbf{X}^* \gets \text{num} / \text{den}$
    \State $\mathbf{s}^* \gets \text{""}$
    \While{$|x - \mathbf{X}^*| > \delta_x$}
        \If{$\mathbf{X}^* < x$} 
            \State $\text{left} \gets (\text{num}, \text{den})$
            \State $\mathbf{s}^* \gets \mathbf{s}^* + \text{'1'}$
        \Else 
            \State $\text{right} \gets (\text{num}, \text{den})$
            \State $\mathbf{s}^* \gets \mathbf{s}^* + \text{'0'}$
        \EndIf
        \State $\text{num} \gets \text{left}_{num} + \text{right}_{num}$
        \State $\text{den} \gets \text{left}_{den} + \text{right}_{den}$
        \State $\mathbf{X}^* \gets \text{num} / \text{den}$
    \EndWhile
    \State $\epsilon_x \gets \mathbf{X}^* - x$
    \State \Return $\mathbf{X}^*, \mathbf{s}^*, \epsilon_x$
\EndFunction
\end{algorithmic}

\subsection{The Harmonic Oscillator Simulation}

\begin{equation}
(\mathbf{Q}^*, \mathbf{s}^*_q, \epsilon_q) \gets \texttt{erase}(q_?, \mathcal{P}^{-1})
\end{equation}
\begin{equation}
(\mathbf{P}^*, \mathbf{s}^*_p, \epsilon_p) \gets \texttt{erase}(p_?, \mathcal{Q}^{-1})
\end{equation}


\section{Foundational Manuscripts}

\begin{itemize}
    \item \textbf{Geometry}
    \begin{itemize}
        \item Heisenberg \cite{Heisenberg1927}
        \item de Gosson \cite{DeGosson2003}
    \end{itemize}
    
    \item \textbf{Thermodynamics}
    \begin{itemize}
        \item Landauer \cite{Landauer1961}
        \item Plenio and Vitelli \cite{PlenioVitelli2001}
        \item Margolus and Levitin \cite{MargolusLevitin1998}
    \end{itemize}
    
    \item \textbf{Methodology}
    \begin{itemize}
        \item Stern \cite{Stern1858}
        \item Brocot \cite{Brocot1861}
        \item Graham, Knuth, and Patashnik \cite{ConcreteMath}
        \item Stolzenburg \cite{Stolzenburg2015}
        \item Aiylam \cite{Aiylam2013}
    \end{itemize}
\end{itemize}

\bibliography{main} 
\end{document}
