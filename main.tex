\documentclass[%
 aps, reprint, prl, amsmath,amssymb
]{revtex4-2}

\usepackage{graphicx}% Include figure files
\graphicspath{{figures/}} 
\usepackage{dcolumn}% Align table columns on decimal point
\usepackage{bm}% bold math
\usepackage{orcidlink}
\usepackage{mathtools}

% -------------- Watermark ----------------
\usepackage[style=iso]{datetime2}
\usepackage{FiraMono} 
\usepackage{tikz}
\usepackage{transparent} 

% Use shipout/background to ensure it stays behind text/figures
\AddToHook{shipout/background}{
  \begin{tikzpicture}[remember picture, overlay]
    \node [
      rotate=55,
      scale=1.2,
      text opacity=0.15,      
      color=gray!50,          
      font=\fontfamily{FiraMono-TLF}\selectfont\bfseries,
      align=center
    ] at (current page.center) {
        {\fontsize{60}{70}\selectfont WORKING DRAFT} \\ [0.8cm]
        {\Huge \DTMnow} \\
        {\Huge doi.org/10.17605/OSF.IO/EV8H6} \\ [0.8cm]
        {\Large Computational Complex Systems Laboratory} \\
        {\Large Homey Music, Detroit, USA} \\ [0.2cm]
         {\transparent{0.25}\includegraphics[width=1.0cm]{logo.png}}
    };
  \end{tikzpicture}
}
% -----------------------------------------


\DeclareMathOperator*{\argmin}{argmin}
\begin{document}

\title{Information Erasure Inside Action Quanta}

\begin{abstract}
Quantum microstates are operationally indistinguishable within finite-support Heisenberg cells. Inside conjugate pairs of these de Gosson action quanta, governed by the principle of least action, Landauer information erasure precludes the classical continuum, necessitating discrete configurations and the emergence of quantum mechanics.
\end{abstract}

\author{Brian S. Mulloy\orcidlink{0000-0002-1803-3172}}
\email{brian@homeymusic.com}
\affiliation{Computational Complex Systems Laboratory\\Homey Music, Corktown, Detroit, MI, USA}
\date{\today}

\maketitle

\section{Figures}

\begin{figure}[t!]
    \centering
    \includegraphics[width=\linewidth]{erase.pdf}
\caption{\textbf{Information erasure inside conjugate action quanta} (schematic). 
System boundaries $\Delta q$ and $\Delta p$ determine the total action $A = \pi \hbar$, while the reciprocal relations $\delta_q = 2\hbar / \Delta p$ and $\delta_p = 2\hbar / \Delta q$ characterize the symplectic squeezing of the conjugate pair of action quanta: $A_q = \frac{\pi}{4} \delta_q \Delta p = \frac{\pi \hbar}{2}$ and $A_p = \frac{\pi}{4} \Delta q \delta_p = \frac{\pi \hbar}{2}$. The variables $\epsilon_q$ and $\epsilon_p$ represent the erasure distances from the microstate $\mu_{q,p}$ with binary encoding $s_\mu$ to the macrostate $(q,p)$ with the minimal binary encoding $\mathbf{s^*}$, where bit-length $|\mathbf{s^*}| \le |s_\mu|$; the physical scales satisfy $A = A_q + A_p = \pi \hbar$, $\epsilon_q \le \delta_q \le \Delta q$, and $\epsilon_p \le \delta_p \le \Delta p$.}

\label{fig:erase}
\end{figure}

\section{Equations}

\subsection{The Information-Action Limit}

The action available for information erasure inside a conjugate pair of finite-support Heisenberg cells, de Gosson action quanta, is
\begin{equation}
\label{eq:InfoActionLimit}
b E \tau \le \pi \hbar ,
\end{equation}
where $b = b_q + b_p = |\mathbf{s}^*_q| + |\mathbf{s}^*_p|$ is the count of bits in the minimal binary encoding $\mathbf{s}^*_q, \mathbf{s}^*_p$ of the physical state $(q,p)$, $E = k_B T \ln 2$ is the Landauer erasure energy, and $\tau$ is the erasure time. Here $\pi \hbar$ represents a conjugate pair of de Gosson action quanta, defined as the minimal phase-space area of finite support; unlike statistical variance $\sigma_q \sigma_p$, the boundaries $\Delta q$ and $\Delta p$ define the hard Heisenberg limits where states are operationally indistinguishable \cite{DeGosson2003}.

\subsection{Information Erasure Inside the Conjugate Action Quanta}

Landauer erasure within a conjugate pair of de Gosson action quanta is bounded by $A_q + A_p = \pi \hbar$, where 
\begin{subequations}
\begin{equation}
\label{eq:SymplecticBoundsq}
A_q = \frac{\pi}{4} \delta_q \Delta p = \frac{\pi \hbar}{2} 
\end{equation}
\begin{equation}
\label{eq:SymplecticBoundsp}
A_p = \frac{\pi}{4}\Delta q \delta_p = \frac{\pi \hbar}{2} 
\end{equation}
\end{subequations} 
yielding the reciprocal relations
\begin{subequations}
\begin{equation}
\label{eq:deltaq}
\delta_q= \frac{2 \hbar}{\Delta p}
\end{equation}
\begin{equation}
\label{eq:deltap}
\delta_p = \frac{2 \hbar}{\Delta q}.
\end{equation}
\end{subequations} 
The operationally indistinguishable microstates $\{\mu_q, \mu_p\}$ are irreversibly erased to macrostates $(q,p)$, where the erasure residues $\epsilon_q$ and $\epsilon_p$ represent the loss of resolution:
\begin{subequations}
\begin{equation}
\label{eq:epsilonq}
\epsilon_q = \mu_q - q
\end{equation} 
and
\begin{equation}
\label{eq:epsilonp}
\epsilon_p = \mu_p - p
\end{equation}
\end{subequations}
Here $|\epsilon_q| \le \delta_q$ and $|\epsilon_p| \le \delta_p$ are encoded with the action available in each orthogonal quanta, where the phase-space coordinates map to the minimal binary addresses $(q,p) \mapsto \mathbf{s}^*_q, \mathbf{s}^*_p$.


\subsection{Thermodynamics and Geometry}

These equations relate by

\begin{equation}
\label{eq:ThermoGeo}
A_L \le A \le A_{d\Gamma} = \pi \hbar,
\end{equation}

where $A_L \coloneqq b E \tau$, $A = A_q + A_p$, and $A_{d\Gamma} \coloneqq \pi \hbar$. 

\subsection{Fine Structure}

There is a self-similar hierarchy $|\epsilon_q| \le \delta_q \le \Delta q$ where $\alpha$ is the  ratio between three nested physical scales:
\begin{itemize}
        \item \textbf{Erasure $|\epsilon_q|$:} The ``micro'' range where microstates $\mu_q$ are collapsed.
        \item \textbf{Squeezed $\delta_q$:} The ``meso'' range of the squeezed Heisenberg cell boundary.
        \item \textbf{Quantum $\Delta q$:} The ``macro'' range of the Hamiltonian support.
    \end{itemize}
    \textit{Insight:} Each level of the hierarchy ``erases'' the same proportion of information ($\alpha$) to move to the next level of coarse-graining.

\begin{equation}
\label{eq:alpha1}
\alpha_\epsilon = \frac{|\epsilon_q|}{\delta_q} = \frac{|\epsilon_q| \Delta p}{2 \hbar}
\end{equation} where $\epsilon_q \coloneqq \mu_q - q$ is the erasure distance and $\delta_q \coloneqq\frac{2 \hbar}{\Delta p}$ is the quantum of action coordinate bound for $A_q$. The ratio of the coordinate quantum of action bound to the overall Hamiltonian (akin to $\lambda_e/a_0$) is

\begin{equation}
\label{eq:alpha2}
\alpha_\delta =  \frac{\delta_q}{\Delta q} = \frac{2 \hbar}{\Delta q \Delta p}.
\end{equation} And the ratio of the erasure distance to the overall Hamiltonian (akin to $r_e/a_0$)

\begin{equation}
\label{eq:alpha3}
\alpha_\Delta^2 = \frac{|\epsilon_q|}{\Delta q}.
\end{equation}

All together we have \begin{equation}
\label{eq:alpha_combined}
|\epsilon_q| = \alpha \delta_q = \alpha^2 \Delta q
\end{equation}

\section{Normalized Equations}

\subsection{Information Erasure Inside Action Quanta}

Landauer erasure within a conjugate pair of Heisenberg cells is bounded by:
\begin{equation}
\label{eq:SymplecticBounds}
|q_? - \mathbf{Q}^*| \le \mathcal{P}^{-1}, 
\quad 
|p_? - \mathbf{P}^*| \le \mathcal{Q}^{-1},
\end{equation}
where the operationally indistinguishable microstates are represented as $\mu_q \coloneqq q_0 q_?$ and $\mu_p \coloneqq p_0 p_?$, while the macrostates $\mathbf{Q}^*$ and $\mathbf{P}^*$, encoded with the action available in each orthogonal quanta, are defined as $q \coloneqq q_0 \mathbf{Q}^*$ and $p \coloneqq p_0 \mathbf{P}^*$. The maximal extent and maximal momentum of the system are represented as $\Delta q \coloneqq q_0 \mathcal{Q}$ and $\Delta p \coloneqq p_0 \mathcal{P}$ 

\paragraph{Referencing Example:}
The symplectic squeezing is captured by Eq.~(\ref{eq:SymplecticBounds}).

\section{Foundational Manuscripts}

\begin{itemize}
    \item \textbf{Geometry}
    \begin{itemize}
        \item Heisenberg \cite{Heisenberg1927}
        \item de Gosson \cite{DeGosson2003}
    \end{itemize}
    
    \item \textbf{Thermodynamics}
    \begin{itemize}
        \item Landauer \cite{Landauer1961}
        \item Plenio and Vitelli \cite{PlenioVitelli2001}
        \item Margolus and Levitin \cite{MargolusLevitin1998}
    \end{itemize}
    
    \item \textbf{Methodology}
    \begin{itemize}
        \item Stern \cite{Stern1858}
        \item Brocot \cite{Brocot1861}
        \item Graham, Knuth, and Patashnik \cite{ConcreteMath}
        \item Stolzenburg \cite{Stolzenburg2015}
        \item Aiylam \cite{Aiylam2013}
    \end{itemize}
\end{itemize}

\bibliography{main} 
\end{document}
