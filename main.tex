\documentclass[%
 aps, reprint, prl, amsmath,amssymb
]{revtex4-2}

\usepackage{graphicx}% Include figure files
\graphicspath{{figures/}} 
\usepackage{dcolumn}% Align table columns on decimal point
\usepackage{bm}% bold math
\usepackage{orcidlink}
\usepackage{mathtools}
\usepackage{algorithmic}
\usepackage{algpseudocode}

% -------------- Watermark ----------------
\usepackage[style=iso]{datetime2}
\usepackage{FiraMono} 
\usepackage{tikz}
\usepackage{transparent} 

% Use shipout/background to ensure it stays behind text/figures
\AddToHook{shipout/background}{
  \begin{tikzpicture}[remember picture, overlay]
    \node [
      rotate=55,
      scale=1.2,
      text opacity=0.15,      
      color=gray!50,          
      font=\fontfamily{FiraMono-TLF}\selectfont\bfseries,
      align=center
    ] at (current page.center) {
        {\fontsize{60}{70}\selectfont WORKING DRAFT} \\ [0.8cm]
        {\Huge \DTMnow} \\
        {\Huge doi.org/10.17605/OSF.IO/EV8H6} \\ [0.8cm]
        {\Large Computational Complex Systems Laboratory} \\
        {\Large Homey Music, Detroit, USA} \\ [0.2cm]
         {\transparent{0.25}\includegraphics[width=1.0cm]{logo.png}}
    };
  \end{tikzpicture}
}
% -----------------------------------------


\DeclareMathOperator*{\argmin}{argmin}
\begin{document}

\title{Information Erasure Inside Action Quanta}

\begin{abstract}
Quantum microstates are operationally indistinguishable within finite-support Heisenberg cells. Inside conjugate pairs of these de Gosson action quanta, governed by the principle of least action, Landauer information erasure precludes the classical continuum, necessitating discrete configurations and the emergence of quantum mechanics.
\end{abstract}

\author{Brian S. Mulloy\orcidlink{0000-0002-1803-3172}}
\email{brian@homeymusic.com}
\affiliation{Computational Complex Systems Laboratory\\Homey Music, Corktown, Detroit, MI, USA}
\date{\today}

\maketitle

\section{Figures}

\begin{figure}[t!]
    \centering
    \includegraphics[width=\linewidth]{erase.pdf}
\caption{\textbf{Information erasure inside conjugate action quanta} (schematic). 
System boundaries $\Delta q$ and $\Delta p$ determine the total action $A = \pi \hbar$, while the reciprocal relations $\delta_q = 2\hbar / \Delta p$ and $\delta_p = 2\hbar / \Delta q$ characterize the symplectic squeezing of the conjugate pair of action quanta: $A_q = \frac{\pi}{4} \delta_q \Delta p = \frac{\pi \hbar}{2}$ and $A_p = \frac{\pi}{4} \Delta q \delta_p = \frac{\pi \hbar}{2}$. The variables $\epsilon_q$ and $\epsilon_p$ represent the erasure distances from the microstate $\mu_{q,p}$ with binary encoding $s_\mu$ to the macrostate $(q,p)$ with the minimal binary encoding $\mathbf{s^*}$, where bit-length $|\mathbf{s^*}| \le |s_\mu|$; the physical scales satisfy $A = A_q + A_p = \pi \hbar$, $\epsilon_q \le \delta_q \le \Delta q$, and $\epsilon_p \le \delta_p \le \Delta p$.}

\label{fig:erase}
\end{figure}









\section{Equations}

\subsection{The Information-Action Limit}

The action available for information erasure within a conjugate pair of finite-support Heisenberg cells, the action quanta, is
\begin{equation}
\label{eq:InfoActionLimit}
b E \tau \le \pi \hbar ,
\end{equation}
where $b = b_q + b_p = |\mathbf{s}^*_q| + |\mathbf{s}^*_p|$ is the count of bits in the minimal binary encoding $\mathbf{s}^*_q, \mathbf{s}^*_p$ of the physical state $(q,p)$, $E = k_B T \ln 2$ is the Landauer erasure energy, and $\tau$ is the erasure time. The total action $\pi \hbar$ is the minimal phase-space area of finite support; unlike statistical variance $\sigma_q \sigma_p$, the boundaries $\Delta q$ and $\Delta p$ define the hard Heisenberg limits where states are operationally indistinguishable \cite{DeGosson2003}.

\section{Information Erasure Inside the Conjugate Action Quanta}

\subsection{The General Form: Conjugate Action Quanta}

The physical action available for information erasure is constrained within a conjugate pair of action quanta. The total action, $A_q + A_p = \pi \hbar$, partitions between the spatial and momentum quanta according to the relative geometric proportions of the phase-space supports, $\Delta q$ and $\Delta p$.

\textit{Action Partition:}
\begin{equation}
\label{eq:GeneralPartition}
A_q = \frac{\pi \hbar}{1 + \left ( \frac{\Delta p / p_0}{\Delta q / q_0}\right )^2}, \quad A_p = \frac{\pi \hbar}{1 + \left (\frac{\Delta q/q_0}{\Delta p/p_0}\right )^2 }
\end{equation}

\textit{Symplectic Capacities:}
\begin{equation}
\label{eq:GeneralCapacities}
\delta_q = \frac{4 A_q}{\pi \Delta p}, \quad \delta_p = \frac{4 A_p}{\pi \Delta q}
\end{equation}

\textit{Discrete Tiling:}
\begin{equation}
\label{eq:GeneralTiling}
N_q = \max\left(1, \left\lfloor \frac{\Delta p/p_0}{\Delta q / q_0} \right\rfloor\right), \quad N_p = \max\left(1, \left\lfloor \frac{\Delta q/q_0}{\Delta p/p_0} \right\rfloor\right)
\end{equation}

The supports $\Delta p$ and $\Delta q$ define the macrostate boundaries, while the minor diameters $\delta_q$ and $\delta_p$ define the resolution limit—the threshold of the hard Heisenberg limits where microstates become operationally indistinguishable.

\subsection{The Harmonic Oscillator: Symmetrical Quanta}

The Harmonic Oscillator is a state of dynamical equilibrium. In the ground state, the potential and kinetic energies are balanced, resulting in symmetric supports where $\Delta q = \Delta p$.

\textit{Action Partition:}
\begin{equation}
A_q = \frac{\pi}{4} \delta_q \Delta p = \frac{\pi \hbar}{2}, \quad A_p = \frac{\pi}{4} \Delta q \delta_p = \frac{\pi \hbar}{2}
\end{equation}

\textit{Symplectic Capacities:}
\begin{equation}
\delta_q = \frac{2 \hbar}{\Delta p}, \quad \delta_p = \frac{2 \hbar}{\Delta q}
\end{equation}

\textit{Discrete Tiling:}
\begin{equation}
N_q = 1, \quad N_p = 1
\end{equation}

\subsection{The Particle in a Box: Asymmetrical Quanta}

In a box of width $L$, the spatial support is fixed ($\Delta q = L$). This physical constraint forces the conjugate pair into a geometric asymmetry as the momentum support $\Delta p$ increases relative to the spatial scale $q_0$.

\textit{Action Partition:}
\begin{equation}
A_q = \frac{\pi \hbar}{1 + \left ( \frac{\Delta p/p_0}{L/q_0}\right )^2}, \quad A_p = \frac{\pi \hbar}{1 + \left (\frac{L/q_0}{\Delta p / p_0} \right )^2}
\end{equation}

\textit{Symplectic Capacities:}
\begin{equation} 
\delta_q = \frac{4 \hbar}{\Delta p \left( 1 + \left[ \frac{\Delta p/p_0}{L/q_0} \right]^2 \right)}, \quad \delta_p = \frac{4 \hbar}{L \left( 1 + \left[ \frac{L/q_0}{\Delta p/p_0} \right]^2 \right)} 
\end{equation}

\textit{Discrete Tiling:}
\begin{equation}
N_q = \max\left(1, \left\lfloor \frac{\Delta p/p_0}{L/q_0} \right\rfloor\right), \quad N_p = 1
\end{equation}

As $\Delta p$ increases, the momentum resolution remains at its fundamental limit, while the spatial support $L$ is resolved by an increasing count of tiles defined by the squeezed spatial capacity $\delta_q$.

\subsection{The Free Particle: Conjugate Action Redistribution}

In the Free Wave Packet, the momentum support remains a constant of motion $p_0$, while the spatial support $\Delta q(t)$ increases linearly. This expansion is a continuous redistribution of action within the conjugate pair. 

\textit{Action Partition:}
\begin{equation}
A_q(t) = \frac{\pi \hbar}{1 + \left ( \frac{1}{\Delta q(t) / q_0}\right )^2}, \quad A_p(t) = \frac{\pi \hbar}{1 + \left ( \Delta q(t)/q_0 \right )^2 }
\end{equation}

\textit{Symplectic Capacities:}
\begin{equation}
\delta_q(t) = \frac{4 A_q(t)}{\pi p_0}, \quad \delta_p(t) = \frac{4 A_p(t)}{\pi \Delta q(t)}
\end{equation}

\textit{Discrete Tiling:}
\begin{equation}
N_q(t) = \max\left(1, \left\lfloor \frac{1}{\Delta q(t) / q_0} \right\rfloor\right), \quad N_p(t) = \max\left(1, \left\lfloor \Delta q(t) / q_0 \right\rfloor\right)
\end{equation}

The total action budget is conserved as the spatial dimension draws action from its conjugate momentum partner, allowing the state to maintain its operational structure as it spreads.

\subsection{Bell's Theorem: Angular Quanta}

\begin{equation}
\label{eq:GeneralPartition}
A_\theta = ?, \quad A_L = ? 
\end{equation}

\section{Thermodynamics and Geometry}

These equations relate by

\begin{equation}
\label{eq:ThermoGeo}
A_L \le A \le A_{d\Gamma} = \pi \hbar,
\end{equation}

where $A_L \coloneqq b E \tau$, $A = A_q + A_p$, and $A_{d\Gamma} \coloneqq \pi \hbar$. 

\subsection{Fine Structure}

There is a self-similar hierarchy $|\epsilon_q| \le \delta_q \le \Delta q$ where $\alpha$ is the  ratio between three nested physical scales:
\begin{itemize}
        \item \textbf{Erasure $|\epsilon_q|$:} The ``micro'' range where microstates $\mu_q$ are collapsed.
        \item \textbf{Squeezed $\delta_q$:} The ``meso'' range of the squeezed Heisenberg cell boundary.
        \item \textbf{Quantum $\Delta q$:} The ``macro'' range of the Hamiltonian support.
    \end{itemize}
    \textit{Insight:} Each level of the hierarchy ``erases'' the same proportion of information ($\alpha$) to move to the next level of coarse-graining.

\begin{equation}
\label{eq:alpha1}
\alpha_\epsilon = \frac{|\epsilon_q|}{\delta_q} = \frac{|\epsilon_q| \Delta p}{2 \hbar}
\end{equation} where $\epsilon_q \coloneqq \mu_q - q$ is the erasure distance and $\delta_q \coloneqq\frac{2 \hbar}{\Delta p}$ is the quantum of action coordinate bound for $A_q$. The ratio of the coordinate quantum of action bound to the overall Hamiltonian (akin to $\lambda_e/a_0$) is

\begin{equation}
\label{eq:alpha2}
\alpha_\delta =  \frac{\delta_q}{\Delta q} = \frac{2 \hbar}{\Delta q \Delta p}.
\end{equation} And the ratio of the erasure distance to the overall Hamiltonian (akin to $r_e/a_0$)

\begin{equation}
\label{eq:alpha3}
\alpha_\Delta^2 = \frac{|\epsilon_q|}{\Delta q}.
\end{equation}

All together we have \begin{equation}
\label{eq:alpha_combined}
|\epsilon_q| = \alpha \delta_q = \alpha^2 \Delta q
\end{equation}
\section{Normalized Equations}

\subsection{Normalized Harmonic Oscillator}

Each of the two minimal action quanta are $A_0 = 2 \cdot\frac{\pi}{4} q_0 p_0 = \pi \hbar$. The operationally indistinguishable microstates are represented as $\mu_q \coloneqq q_0 q_?$ and $\mu_p \coloneqq p_0 p_?$, while the macrostates $\mathbf{q}^*$ and $\mathbf{p}^*$, encoded with the action available in each orthogonal quanta, are defined as $q \coloneqq q_0 \mathbf{q}^*$ and $p \coloneqq p_0 \mathbf{p}^*$. The maximal extent and maximal momentum of the system are represented as $\Delta q \coloneqq q_0 \mathcal{Q}$ and $\Delta p \coloneqq p_0 \mathcal{P}$, subject to the support constraints $\mathcal{Q}, \mathcal{P} \ge 1$, which prevent resolution from exceeding the system's unit action scale.

\begin{equation}
\frac{\pi}{4} \delta_q \Delta p = \frac{\pi \hbar}{2}
\end{equation}

\begin{equation}
\frac{\pi}{4} q_0 |q_? - \mathbf{q}^*| p_0 \mathcal{P} = \frac{\pi \hbar}{2}
\end{equation}

\begin{equation}
\label{eq:SymplecticBounds}
|q_? - \mathbf{q}^*| \le \mathcal{P}^{-1}, 
\quad 
|p_? - \mathbf{p}^*| \le \mathcal{Q}^{-1},
\end{equation}

\subsection{Normalized Bell Test}

The angular action quanta are defined by the circular support $A_0 = 2 \cdot\frac{\pi}{4} \theta_0 L_0 = \pi \hbar$, representing a state of rotational equilibrium analogous to the harmonic oscillator ground state. The operationally indistinguishable angular microstates are $\mu_\theta \coloneqq \theta_0 \theta_?$ and $\mu_L \coloneqq L_0 L_?$, while the macrostates $\mathbf{\theta}^*$ and $\mathbf{L}^*$ are defined by the action available in each orthogonal quanta as $\theta \coloneqq \theta_0 \mathbf{\theta}^*$ and $L \coloneqq L_0 \mathbf{L}^*$. Because the de Gosson blob is an isotropic disk, the unit momentum scale is defined by the radius $L_0 = \sqrt{\pi \hbar}$. The maximal angular momentum $\Delta L \coloneqq L_0 \mathcal{L}$ constrains the resolution of the microstate $\theta_?$, where the factor $2/\sqrt{\pi}$ emerges explicitly from the ratio of the disk's diameter $D = 2\sqrt{\hbar}$ to the linear unit scale $L_0$:

\begin{equation}
\frac{D}{L_0} = \frac{2\sqrt{\hbar}}{\sqrt{\pi \hbar}} = \frac{2}{\sqrt{\pi}}
\end{equation}

Substituting this geometric ratio into the support constraint ensures the resolution does not exceed the system's isotropic unit scale:

\begin{equation}
\frac{\pi}{4} \theta_0 |\theta_? - \mathbf{\theta}^*| (\sqrt{\pi \hbar} \mathcal{L}) = \frac{\pi \hbar}{2}
\end{equation}

\begin{equation}
\label{eq:AngularSymplecticBounds}
|\theta_? - \mathbf{\theta}^*| \le \frac{2}{\sqrt{\pi} \mathcal{L}}, 
\quad 
|L_? - \mathbf{L}^*| \le \frac{2}{\sqrt{\pi} \Theta}
\end{equation}

\subsection{Normalized Rigid Rotor}

Landauer erasure for the rigid rotor within conjugate action quanta is bounded by the phase-dependent resolution limits:
\begin{equation}
\label{eq:RotorBounds}
|q_? - \mathbf{Q}^*| \le \frac{4 \cos^2 \phi}{\pi |\sin \phi|}, 
\quad 
|p_? - \mathbf{P}^*| \le \frac{4 \sin^2 \phi}{\pi |\cos \phi|},
\end{equation}
where the operationally indistinguishable microstates are represented as $\mu_q \coloneqq R q_?$ and $\mu_p \coloneqq p_\omega p_?$, while the macrostates $\mathbf{Q}^*$ and $\mathbf{P}^*$, encoded with the action partitioned by the phase angle $\phi$, are defined as $q \coloneqq R \mathbf{Q}^*$ and $p \coloneqq p_\omega \mathbf{P}^*$. 

Physical accessibility is maintained only for non-vanishing conjugate supports; specifically, spatial resolution requires $|\sin \phi| > 0$ while momentum resolution requires $|\cos \phi| > 0$. The divergence of these capacities represents the threshold of total information erasure, mirroring the Heisenberg floor ($\mathcal{P}, \mathcal{Q} \ge 1$) where the system reaches the limit of a single unit action quantum.

\subsection{Normalized Bell's Theorem}

Landauer erasure for the shared action state is bounded by the local resolution limits of Alice and Bob:
\begin{equation}
\label{eq:BellBounds}
|\theta_{?\alpha} - \mathbf{\Theta}_\alpha^*| \le \frac{4 \cos^2 \alpha}{\pi |\sin \alpha|}, 
\quad 
|\theta_{?\beta} - \mathbf{\Theta}_\beta^*| \le \frac{4 \sin^2 \beta}{\pi |\cos \beta|},
\end{equation}
where $\theta_{?\alpha}, \theta_{?\beta}$ are microstates erased to macrostates $\mathbf{\Theta}_\alpha^*, \mathbf{\Theta}_\beta^*$. At the unit action scale, the available budget lacks the capacity to encode the precise value of the erasure distances:
\begin{equation}
\label{eq:BellSign}
\epsilon_\alpha = \theta_{?\alpha} - \mathbf{\Theta}_\alpha^*, 
\quad 
\epsilon_\beta = \theta_{?\beta} - \mathbf{\Theta}_\beta^*.
\end{equation}

Consequently, the measurement outcome is truncated to a single bit of information:
\begin{equation}
\theta_\alpha = \text{sgn} (\epsilon_{\alpha}), \quad \theta_\beta = \text{sgn} (\epsilon_\beta).
\end{equation}

This binary truncation is a physical requirement of the $\pi \hbar$ system budget. With only two available bits for the joint state, the universe resolves the local sign of the partition while erasing the higher-order geometric details.

Physical accessibility is determined strictly by the local setting; resolution requires non-vanishing conjugate support ($|\sin \alpha| > 0, |\cos \beta| > 0$).


\section{Numerical Simulations}

\begin{algorithmic}
\Function{erase}{$x, \delta_x$}
    \State $\text{left} \gets (-1, 0), \quad \text{right} \gets (1, 0)$
    \State $\text{num} \gets 0, \quad \text{den} \gets 1$
    \State $\mathbf{X}^* \gets \text{num} / \text{den}$
    \State $\mathbf{s}^* \gets \text{""}$
    \While{$|x - \mathbf{X}^*| > \delta_x$}
        \If{$\mathbf{X}^* < x$} 
            \State $\text{left} \gets (\text{num}, \text{den})$
            \State $\mathbf{s}^* \gets \mathbf{s}^* + \text{'1'}$
        \Else 
            \State $\text{right} \gets (\text{num}, \text{den})$
            \State $\mathbf{s}^* \gets \mathbf{s}^* + \text{'0'}$
        \EndIf
        \State $\text{num} \gets \text{left}_{num} + \text{right}_{num}$
        \State $\text{den} \gets \text{left}_{den} + \text{right}_{den}$
        \State $\mathbf{X}^* \gets \text{num} / \text{den}$
    \EndWhile
    \State $\epsilon_x \gets \mathbf{X}^* - x$
    \State \Return $\mathbf{X}^*, \mathbf{s}^*, \epsilon_x$
\EndFunction
\end{algorithmic}

\subsection{The Harmonic Oscillator Simulation}

\begin{equation}
(\mathbf{Q}^*, \mathbf{s}^*_q, \epsilon_q) \gets \texttt{erase}(q_?, \mathcal{P}^{-1})
\end{equation}
\begin{equation}
(\mathbf{P}^*, \mathbf{s}^*_p, \epsilon_p) \gets \texttt{erase}(p_?, \mathcal{Q}^{-1})
\end{equation}

\section{Topological Node Extraction via Deterministic Hysteresis}

To extract the effective quantum number $n$ from the deterministic erasure density $\rho(q)$, we employ a hysteresis-based topological filter. Unlike stochastic models, our density function is fully deterministic and repeatable; however, the discrete nature of the Stern-Brocot transform introduces high-frequency fine-structure modulations (quantization artifacts) superimposed on the macroscopic wavefunction.

To rigorously distinguish true topological nodes from these fine-scale artifacts, we implement a symmetric \textit{Schmitt trigger} detection logic. We define a scale-adaptive prominence threshold $T_{h}$ derived from the signal's mean amplitude:

\begin{equation}
    T_{h} = k \sqrt{\langle \rho \rangle}
\end{equation}

where $k=2$ sets the rejection floor for quantization ripples. This square-root scaling ensures that the detector's sensitivity adapts to the signal intensity, maintaining robust topological identification across orders of magnitude in momentum space.

The algorithm scans the density $\rho(q)$ outward from the potential center ($|q| \to \infty$). A node is registered if and only if the density executes a full hysteresis loop: descending into a local minimum and subsequently recovering by an amplitude $\Delta \rho \ge T_{h}$. This logic effectively filters out the deterministic "texture" of the number-theoretic lattice, isolating the fundamental oscillatory modes of the physical state.

\section{Foundational Manuscripts}

\begin{itemize}
    \item \textbf{Geometry}
    \begin{itemize}
        \item Heisenberg \cite{Heisenberg1927}
        \item de Gosson \cite{DeGosson2003}
    \end{itemize}
    
    \item \textbf{Thermodynamics}
    \begin{itemize}
        \item Landauer \cite{Landauer1961}
        \item Plenio and Vitelli \cite{PlenioVitelli2001}
        \item Margolus and Levitin \cite{MargolusLevitin1998}
    \end{itemize}
    
    \item \textbf{Methodology}
    \begin{itemize}
        \item Stern \cite{Stern1858}
        \item Brocot \cite{Brocot1861}
        \item Graham, Knuth, and Patashnik \cite{ConcreteMath}
        \item Stolzenburg \cite{Stolzenburg2015}
        \item Aiylam \cite{Aiylam2013}
    \end{itemize}
\end{itemize}

\bibliography{main} 
\end{document}
